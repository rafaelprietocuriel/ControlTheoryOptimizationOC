\section{Model class}
In the current version of \OCMAT\ only the class \lstinline+stdocmodel+ is implemented. The implementation of further classes and extensions are planned, e.g. multi-stage \OCPRO s or space distributed \OCPRO s.

\subsection{Standard Model Class (\texorpdfstring{\lstinline+stdocmodel+}{stdocmodel})}\index{Classes!stdocmodel|indexbf}
Any instance \lstinline+ocObj+ of the class \lstinline+stdocmodel+ represents a model of type \cref{eq:simple_simple_opt_pro} specified for a given set of parameter values. To add a new model an initialization file has to be provided and processed, for examples see \cref{sec:InitializationMoM,sec:InitializationCRM}. 

This step only has to be done once. During the initialization of a new model the necessary \MATL\ files for a numerical analysis are created. These files are moved into the default model folder, e.g. for the model \MoM\ \lstinline+ocmat\model\usermodel\mom+. Additionally a structure \lstinline+ocStruct+ is generated and stored in \lstinline+ocmat\model\usermodel\mom\data\momModelDataStructure.mat+. This structure contains all provided and/or derived information of the model (e.g. first order necessary conditions, etc.) that is necessary to create the \MATL\ files. Usually the user need not access this structure directly and only  the experienced user should change its entries.\footnote{That manual changes of the \lstinline+ocStruct+ structure become active the function \lstinline+makefile4ocmat(ocStruct)+ has to be recalled. Then the model files are created with the changes made in \lstinline+ocStruct+.}

To instantiate a specific model, e.g., model \MoM\, call \lstinline+m=ocStruct('mom')+ (for alternative calls see the command description on \cpageref{cmd:stdocmodel}). This creates an object \lstinline+m+ of class \lstinline+stdocmodel+. This class consists of the two fields \lstinline+Model+ and \lstinline+Result+, where the latter is empty. The field \lstinline+Model+
\begin{matlab}
>> m.Model
ans = 
             modeltype: 'standardmodel'
             modelname: 'mom'
              variable: [1x1 struct]
            constraint: [1x1 struct]
             objective: [1x1 struct]
      optimizationtype: 'max'
             parameter: [1x1 struct]
           description: {'a model of moderation, simple one state model with indifference solutions'}
                   arc: [1x1 struct]
    pontryaginfunction: [1x1 struct]
\end{matlab}
consists of the structure \lstinline+ocStruct+, without the field \lstinline+foc+ for the first order necessary conditions. This field is not included since the first order necessary optimality conditions are only needed for the \MATL\ file generation during the initialization process of the model. Each field can be addressed as fields of a usual structure, e.g.
\begin{matlab}
>> m.Model.parameter
ans = 
    variable: [1x1 struct]
         num: 2
>> m.Model.parameter.variable
ans = 
    r: 1
    c: 2
\end{matlab}
But for important fields specific commands are provided that allow an easy access, e.g.
\begin{matlab}
>> [val name]=parametervalue(m)
val =
     1     2
name = 
    'r'    'c'
>> modelname(m)
ans =
mom
\end{matlab}
An instance generated by the class constructor \lstinline+stdocmodel+ uses the default parameter values of the initialization file. The parameter values can be changed using \lstinline+changeparametervalue+ (see \cpageref{cmd:changeparametervalue}).

As shown in the examples in \cref{sec:numanalysis,sec:numanalysis2D} (numerical) results can be stored (\hyperref[cmd:store]{\lstinline+store+}), the instance can be saved (\hyperref[cmd:save]{\lstinline+save+}) and loaded (\hyperref[cmd:load]{\lstinline+load+})

\section{Solution classes}
Most of the relevant data derived during the analysis of an \OCPRO\ are time depending functions (mainly solutions of the canonical system). To cover these data the class of \lstinline+octrajectory+ is implemented. In general this represents the solution of an \ODE. Based on this class further classes are derived. Specific trajectories are equilibria, limit cycles (class \lstinline+dynprimitive+), or solutions converging to such limit sets (class \lstinline+ocasymptotic+), multiple solutions of an \OCPRO\ are represented by the class \lstinline+ocmultipath+. Additionally, curves/manifolds in the state space, etc. are represented by specific points (class \lstinline+occurve+). In this section the main features of these classes are presented.

\subsection{Time path (\texorpdfstring{\lstinline+octrajectory+}{octrajectory})}\index{Classes!octrajectory|indexbf}
The fields of an instance \lstinline+ocTrj+ of the class \lstinline+octrajectory+ are
\begin{matlab}
                 x: [ row vector (length N) ]
                 y: [ matrix (size n x N) ]
            arcarg: [ scalar or integer row vector (length a) ]
       arcposition: [ integer matrix (size 2 x a) ]
       arcinterval: [ row vector (length a+1) ]
                x0: [ scalar ]
     linearization: [ 3D matrix (size n x n x N) / empty ]
            solver: [ string / empty ]
       timehorizon: [ scalar (double or inf) / empty ]
         modelname: [ string / empty ]
    modelparameter: [ row vector / empty]
        solverinfo: [ struct / empty ]
          userinfo: [ struct / empty ]
     violationinfo: [ struct / empty ]
\end{matlab}
where the fields have the following meaning.
\begin{description}
	\item[\texttt ocTrj.x]  Time discretization of the solution path (independent variable). The values are normalized to the interval $[0,a]$, where $a$ is the number of arcs. The switching times between arc $i$ and $i+1$ are normalized to the value $i$ and appear twice (see the \MATL\ reference for \lstinline+bvp4c+). Thus, for a solution path consisting of multiple arcs ($a>1$) the time variable is of the form
\begin{equation*}
	0<t_1^1<\ldots<t_{n_1}^1<1=1<t_1^2<\ldots t_{n_2}^2<2<\ldots<a-1<t_1^a<\ldots<t_{n_a}^a<a.
\end{equation*}
The values at the double entries of the switching times are therefore the left and right side limits.
\item[\texttt ocTrj.y]Values of the solution path at the time discretization (dependent variable). Usually these are the state and costate values.
\item[\texttt ocTrj.arcarg]The arcidentifier(s) corresponding to the solution path. 
\item[\texttt ocTrj.arcposition]The $i$'th column represents the left and right position of the $i$'th arc in the discretization. 
\item[\texttt ocTrj.arcinterval]Contains the initial time, switching times and end time. 
\item[\texttt ocTrj.x0]Initial time. If it is empty \lstinline+x0+ is set to zero. (Mandatory) 
\item[\texttt ocTrj.linearization]It represents the linearization along the solution path at the actual discretization. (Mandatory) 
\item[\texttt ocTrj.solver]The \BVP\ Method used for calculation. (Mandatory) 
\item[\texttt ocTrj.timehorizon]The actual time horizon of the problem. If it is empty it is assumed to be infinite. (Mandatory) 
\item[\texttt ocTrj.modelname]The name of the underlying model. (Mandatory) 
\item[\texttt ocTrj.modelparameter]The parameter values of the underlying model. (Mandatory) 
\item[\texttt ocTrj.solverinfo]Structure containing information that are set during the continuation process. (Mandatory) 
\item[\texttt ocTrj.userinfo]Structure that can be set by the user. (Mandatory) 
\item[\texttt ocTrj.violationinfo]Structure containing information about possible violations along the solution path. 
\end{description}
For the construction of the class \lstinline+octrajectory+ the values for the fields \lstinline+x+, \lstinline+y+, \lstinline+arcarg+ and \lstinline+arcinterval+ have to be provided, e.g.
\begin{matlab}
>> N1=3;N2=4;trj.x=[linspace(0,1,N1) linspace(1,2,N2)];trj.y=rand(4,N1+N2);
>> trj.arcarg=[0 3];trj.arcinterval=[0 0.2 100];
>> ocTrj=octrajectory(trj)
ocTrj =
ocmatclass: octrajectory
                 x: [0 0.5000 1 1 1.3333 1.6667 2]
                 y: [4x7 double]
            arcarg: [0 3]
       arcinterval: [0 0.2000 100]
       arcposition: [2x2 double]
                x0: 0
     linearization: []
            solver: ''
       timehorizon: []
         modelname: ''
    modelparameter: []
        solverinfo: []
          userinfo: []
     violationinfo: []
>> ocTrj.y
ans =
    0.4654    0.4715    0.1342    0.4586    0.0735    0.0920    0.2984
    0.5301    0.3024    0.9015    0.4939    0.3386    0.2679    0.6116
    0.2642    0.9334    0.2870    0.3783    0.1010    0.4533    0.5771
    0.1398    0.5066    0.2646    0.8587    0.7347    0.9319    0.1815
>> ocTrj.arcposition
ans =
     1     4
     3     7
\end{matlab}
Additionally the corresponding model can be provided
\begin{matlab}
>> m=stdocmodel('fishery2D');
>> ocTrj=octrajectory(trj,m)
ocTrj =
ocmatclass: octrajectory
                 x: [0 0.5000 1 1 1.3333 1.6667 2]
                 y: [4x7 double]
            arcarg: [0 3]
       arcinterval: [0 0.2000 100]
       arcposition: [2x2 double]
                x0: 0
     linearization: []
            solver: ''
       timehorizon: []
         modelname: 'fishery2D'
    modelparameter: [0.0300 0.0500 1 1 1 1 1 0 2]
        solverinfo: []
          userinfo: []
     violationinfo: []
\end{matlab}
\subsubsection{Arcdependent size of \ODE}
To handle problems consisting of constrain combinations with different number of implicit controls or multi-stage models with changing number of states the class  \lsi+octrajectory+ is extended. The number of \ODEs{} can vary with the specific arc. This information has to be included into the \lsi+ocTrj+. We therefore introduce a new class \lsi+ocgtrajectory+. This class inherits the attributes from \lsi+octrajectory+ and has a further field \lsi+odenum+. Additionally the field \lsi+y+ is a cell array.

\subsection{Dynamic primitives (\texorpdfstring{\lstinline+dynprimitive+}{dynprimitive})}\index{Classes!dynprimitive|indexbf}
Dynamic primitives cover the basic objects that can appear in a dynamical system. Mainly (for autonomous systems) these are equilibria and limit cycles. The class \lstinline+dynprimitive+ is derived by simple inheritance from the class \lstinline+octrajectory+, with the additional field \lstinline+period+. Thus, an instance \lstinline+dynPrim+ of the class \lstinline+dynprimitive+ has the fields
\begin{matlab}
          period: [ scalar (double) ]
    octrajectory: [ class octrajectory ]
\end{matlab}
where the fields have the following meaning.
\begin{description}
	\item[\texttt dynPrim.period] For an equilibrium \lstinline+period+ is set to zero. For a periodic solution \lstinline+period+ is set to the period of the solution.
	\item[\texttt dynPrim.octrajectory] This field contains the corresponding trajectory. For an equilibrium this is the constant solution.
\end{description}
	