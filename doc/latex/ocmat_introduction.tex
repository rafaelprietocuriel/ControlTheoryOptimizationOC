%\section{Introduction}

The \OCT\ toolbox is a collection of functions designed to adequately handle optimal control problems in \MATL\TReg\footnote{\MATL\ is a registered trademark of The MathWorks Inc.}. The primary focus of this toolbox lies on discounted, autonomous infinite time horizon models, but it also provides extensions to (non-autonomous) finite time horizon problems. Relying on Pontryagin's \MAXP, at its core is the formulation and solution of boundary value problems in combination with a continuation algorithm. This approach allows a systematic numerical analysis of the problem under varying conditions, like changing initial states or parameter values and therefore the detection of bifurcations of the optimal system. It was \emph{not} the intention to provide a general solver for a wide variety of \OCPRO s, or a very fast solver. The typical situation for the usage of \OCMAT\ is a rather low-dimensional problem in the state space (even though there is no principle limit on the number of states or controls) and analytical functional terms describing the dynamics, objective function and constraints.  

A further advantage of \OCMAT\ is the possibility to derive the first order necessary optimality conditions automatically using the symbolic toolbox of \MATL. This frees the user from this often tedious and error-prone step. Additionally the user also has the possibility to provide the optimality conditions by her/his own, which becomes important if a specific situation has to be considered, that is not included in the standard case. Furthermore, the necessary \MATL\ files are generated automatically, which speeds up the time between model formulation and start of the numerical analysis considerably. 

This automatization process for the file generation relies on the provision of so called template files. These files are pure ASCII files (extension 'oct') and contain variables, which are then replaced by the model specific values. A simple script language also allows the usage of ``if-else'' clauses and ``for'' loops supporting the adaptation on specific needs. An experienced user should therefore be able to write her/his own template files and extending the capabilities of \OCMAT.

Another intention for the development of \OCMAT\ was to create a tool that researchers with only little (mathematical) background knowledge can use for the analysis of standard optimal control models. However, some basic knowledge about the underlying ideas and methods is certainly helpful and for non-standard models even necessary. Introductory books to optimal control theory are numerous, one of these is \cite{grassetal2008} which was written simultaneously to the development of the toolbox. All of the numerical results presented in this book were calculated with \OCMAT. For a first introduction to the used numerical methods the user is referred to \citet{grass2012}. 

But not only problems of mathematical nature were a concern for the design of this toolbox. As a large quantity of information has to be handled, another objective of the toolbox is to ensure that users can efficiently access information interesting not only from a mathematician's point of view, but also information relevant for the economic interpretation of the analyzed model. Closely related is the goal that as much as possible can be handled automatically without losing information, as well as providing ways to influence the outcome. It was also important not to neglect the possibility of further extensions of the toolbox. For all these reason an objective oriented programming approach was used for \OCMAT.

The aim of this manual is to present the main features of the \OCMAT\ toolbox and to provide some guidance to users about how to use it. Additional information are included in the files of the specific functions. Further, on the website\footnote{\httpOCMAT} some demos are provided in order to illustrate \OCMAT's main capabilities.

\section{First Steps for Installation}
\subsection*{System Requirements}
\OCMAT\ requires the correct installation of \MATL\ V.7 or higher. At the moment the derivation of the necessary optimality conditions only work for the Symbolic Toolbox relying on the Maple kernel (\MATL\ V.7.5 and lower). \OCMAT\ accesses by default also the Optimization Toolbox, but this is not a necessary requirement.

For bifurcation related calculations it is suggested that the package \CLMATCONT\ \footnote{download from \httpMATCONT.} is installed on a path accessible to \MATL.

\subsection*{General Preparation}
In order to run \OCMAT, take the following steps:
\begin{enumerate}
\item Make sure you have correctly installed \MATL\ and the necessary toolboxes.
\item Download the \OCMAT-Toolbox and unzip it into any directory where \MATL\ can find it (or add the directory to the \MATL\ path).
\item You can have different versions of \OCMAT\ at the same time, only make sure that \MATL\ has the ``right'' version on its path.
\end{enumerate}

\section{Model Class}
To help the novice in becoming acquainted with the optimal control toolbox (\OCMAT) we start with a simple class of problems. This has the advantage that we can concentrate on the basic concepts without getting lost in technicalities. Thus, we consider 
\begin{subequations}
\label{eq:simple_simple_opt_pro}
\begin{align}
& \max_{u(\cdot)}\int_0^\infty\E^{-rt}g\left(x(t),u(t),\modelpar\right)\,\Dt\label{eq:simple_opt_pro_obj}\\
\text{s.t.}\quad & \dot{x}(t) =\f\left(x(t),u(t),\modelpar\right),\quad t\in[0,\,\infty)\label{eq:simple_opt_pro_dyn}\\
& \mixc\left(x(t),u(t),\modelpar\right)\ge0,\quad t\in[0,\,\infty)\label{eq:simple_opt_pro_mc}\\
\text{with}\quad & x(0) = x_0\label{eq:simple_opt_pro_init}
\end{align}
\end{subequations}
where the state dynamics $f\colon\R^{n+m}\rightarrow\R^n$, the objective function $g\colon\R^{n+m}\rightarrow\R$ and the constraint $\mixc\colon\R^{n+m}\to\R^k$ are assumed to be as often continuously differentiable in their arguments as necessary. Moreover the model is assumed to be nonlinear in the control variable $u\in\R^m$.
